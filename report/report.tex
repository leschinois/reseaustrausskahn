\documentclass[11pt]{article}

\usepackage[french]{babel}
\usepackage[T1]{fontenc}
\usepackage[utf8]{inputenc}

\usepackage{listings}

\lstset{
  basicstyle={\ttfamily}
}

\begin{document}

\title{Impl\'ementation de r\'eseaux de Kahn}

\author{Christophe Cordero, Li-yao Xia}

\maketitle

\section{Composition du projet}

Nous pr\'esentons maintenant les impl\'ementations dans l'ordre
o\`u elles ont \'et\'e \'ecrites, qui est aussi, selon nous, dans l'ordre de technicit\'e croissante.

\section{Impl\'ementation s\'equentielle}

On repr\'esente un processus par le type suivant :

\begin{lstlisting}
  type 'a process = unit -> 'a result
  and 'a result =
    | Proc of 'a process
    | Res of 'a
\end{lstlisting}

Un processus est une fonction qui effectue
un pas {\em atomique} de calcul et :

\begin{enumerate}
  \item Si le calcul n'est pas termin\'e,
    il renvoie la suite du calcul sous forme d'un autre processus {\tt Proc},
  \item Sinon, il renvoie le r\'esultat {\tt Res}.
\end{enumerate}

L'ex\'ecution {\tt run} d'un tel processus est simple :
on fait avancer le calcul jusqu'\`a obtenir un r\'esultat.

\smallskip

Cette repr\'esentation permet alors de simuler un processus {\tt p}
obtenu avec la fonction {\tt doco}, qui repr\'esente l\'execution
parall\`ele d'une liste de processus.

On d\'efinit un pas de calcul atomique pour {\tt p}
comme l'\'execution d'un pas de calcul pour chacun des processus
contenus dans la liste en argument de {\tt doco}.

\smallskip

Les canaux communication sont des files.

\begin{lstlisting}
  type 'a in_port = 'a Queue.t
  type 'a out_port = 'a Queue.t
\end{lstlisting}

On remarquera qu'on repr\'esente le blocage d'un processus {\tt get qi}
en retournant le m\^eme processus quand la file {\tt qi} est vide.

\smallskip

Cette impl\'ementation ne pose pas de probl\`eme majeur
et produit un code tr\`es court.

\section{En utilisant des processus et des tubes}

On repr\'esente un processus par le type le plus simple :

\begin{lstlisting}
  type 'a process = unit -> 'a
\end{lstlisting}

Le parall\'elisme est impl\'ement\'e avec l'utilisation de {\tt Unix.fork}.

\smallskip

Les canaux de communication sont construits \`a l'aide des fonctions
{\tt Unix.pipe},
{\tt Unix.in\_channel\_of\_descr}, {\tt Unix.out\_channel\_of\_descr}.

\begin{lstlisting}
  type 'a in_port = in_channel
  type 'a out_port = out_channel
\end{lstlisting}

Les types {\tt in\_channel} et {\tt out\_channel} permettent d'utiliser
le module {\tt Marshal} directement afin de transf\'erer des valeurs
de type arbitraire.

\smallskip

Cette impl\'ementation ne fonctionne que si le nombre de processus est
born\'e par une constante (nombre de processus que la machine peut g\'erer).

\section{Parall\'elisation sur le r\'eseau en utilisant des {\sl sockets}}

L'architecture propos\'ee est la suivante :

\begin{itemize}
  \item Le programme principal souhaite effectuer un calcul
    {\tt p:'a process}. Pour cela il cr\'ee un serveur,
  \item Des programmes clients peuvent se connecter au serveur pour
    participer au calcul,
    (au moins un tel client est n\'ecessaire au progr\`es)
  \item Les programmes clients peuvent \^etre d\'econnect\'es \`a tout moment
    et le serveur principal doit pouvoir s'adapter \`a ces al\'eas.
\end{itemize}

Pour cela on utilise le type suivant, inspir\'e de l'impl\'ementation
s\'equentielle.

\begin{lstlisting}
  type 'a process =
    | Proc
\end{lstlisting}

\end{document}
