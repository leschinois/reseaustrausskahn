\documentclass[11pt]{article}

\usepackage[french]{babel}
\usepackage[T1]{fontenc}
\usepackage[utf8]{inputenc}

\usepackage{listings}

\lstset{
  basicstyle={\ttfamily}
}

\begin{document}

\title{Implantation de r\'eseaux de Kahn}

\author{Christophe Cordero, Li-yao Xia}

\maketitle

\section{Composition du projet}

\begin{itemize}
  \item {\bf skahn/kahn.ml} Implantation s\'equentielle
  \item {\bf pkahn/kahn.ml} Utilisant des processus et des tubes
  \item {\bf tkahn/\{server,client,kahn\}.ml} Utilisant des threads
  \item {\bf nkahn/kahn.ml} Sur le r\'eseau
  \item {\bf i.ml} Signature + d\'efinitions de fonctions auxiliaires
\end{itemize}

\smallskip

Nous pr\'esentons maintenant les implantations dans l'ordre
o\`u elles ont \'et\'e \'ecrites, qui est aussi, selon nous, dans l'ordre de technicit\'e croissante.

\section{Implantation s\'equentielle}

On repr\'esente un processus par le type suivant :

\begin{lstlisting}
  type 'a process = unit -> 'a result
  and 'a result =
    | Proc of 'a process
    | Res of 'a
\end{lstlisting}

Un processus est une fonction qui effectue
un pas {\em atomique} de calcul et :

\begin{enumerate}
  \item Si le calcul n'est pas termin\'e,
    il renvoie la suite du calcul sous forme d'un autre processus {\tt Proc},
  \item Sinon, il renvoie le r\'esultat {\tt Res}.
\end{enumerate}

L'ex\'ecution {\tt run} d'un tel processus est simple :
on fait avancer le calcul jusqu'\`a obtenir un r\'esultat.

\smallskip

Cette repr\'esentation permet alors de simuler un processus {\tt p}
obtenu avec la fonction {\tt doco}, qui repr\'esente l\'execution
parall\`ele d'une liste de processus.

On d\'efinit un pas de calcul atomique pour {\tt p}
comme l'\'execution d'un pas de calcul pour chacun des processus
contenus dans la liste en argument de {\tt doco}.

\smallskip

Les canaux communication sont des files.

\begin{lstlisting}
  type 'a in_port = 'a Queue.t
  type 'a out_port = 'a Queue.t
\end{lstlisting}

On remarquera qu'on repr\'esente le blocage d'un processus {\tt get qi}
en retournant le m\^eme processus quand la file {\tt qi} est vide.

\smallskip

Cette implantation ne pose pas de probl\`eme majeur
et produit un code tr\`es court.

\section{En utilisant des processus et des tubes}

On repr\'esente un processus par le type le plus simple :

\begin{lstlisting}
  type 'a process = unit -> 'a
\end{lstlisting}

Le parall\'elisme est impl\'ement\'e avec l'utilisation de {\tt Unix.fork}.

\smallskip

Les canaux de communication sont construits \`a l'aide des fonctions
{\tt Unix.pipe},
{\tt Unix.in\_channel\_of\_descr}, {\tt Unix.out\_channel\_of\_descr}.

\begin{lstlisting}
  type 'a in_port = in_channel
  type 'a out_port = out_channel
\end{lstlisting}

Les types {\tt in\_channel} et {\tt out\_channel} permettent d'utiliser
le module {\tt Marshal} directement afin de transf\'erer des valeurs
de type arbitraire.

\smallskip

Cette implantation ne fonctionne que si le nombre de processus est
born\'e par une constante (nombre de processus que la machine peut g\'erer).

\section{Parall\'elisation sur le r\'eseau en utilisant des {\sl sockets}}

L'architecture propos\'ee est la suivante :

\begin{itemize}
  \item Le programme principal souhaite effectuer un calcul
    {\tt p:'a process}. Pour cela il cr\'ee un serveur,
  \item Des programmes clients peuvent se connecter au serveur pour
    participer au calcul,
    (au moins un tel client est n\'ecessaire au progr\`es du calcul)
  \item Les programmes clients peuvent \^etre d\'econnect\'es \`a tout moment
    et le serveur principal doit pouvoir s'adapter \`a ces al\'eas.
  \item Les programmes clients n'ont pas de moyen de parall\'eliser
    des t\^aches. Le parall\'elisme est enti\`erement organis\'e
    par le serveur.
\end{itemize}

Pour cela on utilise le type suivant, inspir\'e de l'implantation
s\'equentielle.

\begin{lstlisting}
  type 'a process = int -> 'a result
  and 'a result =
    | Proc of 'a process
    | Doco of unit process list * 'a process
    | Res of 'a * int
    | U
\end{lstlisting}

Laissons le constructeur U de c\^ot\'e pour l'instant.

Un processus est une fonction qui prend en argument {\tt n : int},
et effectue au maximum {\tt n} pas de calcul avant de retourner:

\begin{enumerate}
  \item Si le calcul n'a pas termin\'e, la suite du calcul.
    Il y a deux cas de figure : ou bien le processus a tout simplement
    consomm\'e tous les pas de calcul disponibles
    et il renvoie un {\tt Proc},
    ou bien il doit lancer des processus en parall\`ele, et il s'arr\^ete
    avant et renvoie la liste des processus parall\`eles,
    ainsi que le calcul une fois que tous ces processus auront termin\'e.
    {\tt Doco (l,p)} traduit le processus
    {\tt bind (doco l) (fun () -> p)}.
  \item Si le calcul est termin\'e, le r\'esultat {\tt Res}
    (avec le nombre de pas non consomm\'es),
\end{enumerate}

Par souci de typage, on impose que la transmission de
t\^aches sous la forme de processus ne se fasse que pour des valeurs
de type {\tt unit process}.

Afin d'\'economiser la bande passante,
le constructeur {\tt U} sert d'alias au r\'esultat {\tt Res ((),\_)}
dont la premi\`ere composante est constante et la deuxi\`eme superflue.
(mais utile pour les calculs interm\'ediaires)

\smallskip

L'objectif vis\'e par cette repr\'esentation est
de permettre aux clients d'interrompre fr\'equemment le calcul.

Les processus constituent des t\^aches.

Les t\^aches sont ex\'ecut\'ees en ordre FIFO.
L'ex\'ecution peut \^etre partielle,
auquel cas la suite du calcul est replac\'ee en fin de liste.

\smallskip

Les avantages recherch\'es sont les suivants :

\begin{itemize}
  \item Le serveur peut sauvegarder les calcul effectu\'es,
    pour avoir \`a en refaire le moins possible en cas de perte d'un client.
  \item M\^eme avec un seul client connect\'e,
    aucun processus ne doit \^etre suspendu ind\'efiniment.
    Si des processus peuvent terminer, notre implantation garantit leur
    terminaison lors de l'ex\'ecution du programme, quels que soient
    le nombre de clients connect\'es et le nombre de t\^aches \`a accomplir.
\end{itemize}

Cela constitue implantation particuli\`erement souple
du mod\`ele de calcul propos\'e.
(au prix d'une performance r\'eduite cependant)

\end{document}
